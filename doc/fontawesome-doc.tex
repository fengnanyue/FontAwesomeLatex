\documentclass{ltxdoc}
%\GetFileInfo{\jobname.sty}
\def\fileversion{4.1.0}
\def\filedate{May 29, 2014}
\usepackage{lmodern}
\usepackage[numbered]{hypdoc}
\usepackage{hologo}
\usepackage{hyperref, xcolor}
\definecolor{myblue}{rgb}{0.22,0.45,0.70}% light blue
\hypersetup{colorlinks=true, linkcolor=myblue, urlcolor=myblue, hyperindex}
\usepackage{longtable, booktabs}
\usepackage{tikz}
\usepackage{xparse, ifthen}
\usepackage{fontawesome}
\EnableCrossrefs
\CodelineIndex
\RecordChanges

\begin{document}
\title{The \textsf{fontawesome} package\\High quality web icons}
\author{%
  Xavier Danaux\thanks{E-mail: \href{mailto:xdanaux@gmail.com}{\tt xdanaux@gmail.com}} (Original \hologo{LaTeX} code)\\%
  Nazar Gerasymchuk\thanks{Github: \href{https://github.com/Troyane}{\tt Troyane}} (New \hologo{LaTeX} code)\\%
  Geoffrey Gouez\thanks{Github: \href{https://github.com/Cicatrice}{\tt Cicatrice}} (New \hologo{LaTeX} code)\\%
  Dave Gandy (font and icons design)}
\date{Version \fileversion, released on \filedate}
\maketitle

\begin{abstract}
The \textsf{fontawesome} package grants access to 439 web-related icons provided by the included \emph{Font Awesome} free font, designed by Dave Gandy and released\footnote{See \url{http://fortawesome.github.com/Font-Awesome} for more details about the font itself} under the open SIL Open Font License\footnote{Available at \url{http://scripts.sil.org/OFL}.}.

This package requires the \textsf{fontspec} package and either the \hologo{Xe}\hologo{(La)TeX} or Lua\hologo{(La)TeX} engine to load the included otf font.
\end{abstract}
\def\currentversion{4.1}
\changes{v4.1.0} {2014/05/29}{Update to match Font Awesome version 4.1.0. New code generation.}
\changes{v3.1.1} {2013/05/10}{Update to match Font Awesome version 3.1.1, with 53 new icons.}
\changes{v3.0.2-1}{2013/03/23}{Bigfix release: corrected the swap of the \cs{text-height} and \cs{text-width} icons.}
\changes{v3.0.2} {2013/03/21}{First public release (version number set to match the included FontAwesome.otf font version).}
\makeatletter
\let\PrintMacroName@original\PrintMacroName
%\let\PrintDescribeMacro\@gobble
%\let\PrintDescribeEnv\@gobble
\let\PrintMacroName\@gobble
%\let\PrintEnvName\@gobble
\begin{macro}{\faTextHeight}
\changes{v3.0.2-1}{2013/03/23}{Corrected binding.}
\end{macro}
\begin{macro}{\faTextWidth}
\changes{v3.0.2-1}{2013/03/23}{Corrected binding.}
\end{macro}
\let\PrintMacroName\PrintMacroName@original
\makeatother

\bigskip

\section{Introduction}
%The \textsf{\jobname} package aims to enable easy access in \hologo{(La)TeX} to high quality icons covering
%\hyperref[section:web_application]{web application},
%\hyperref[section:text_editor]{text editor},
%\hyperref[section:directional]{directional},
%\hyperref[section:video_player]{video player},
%\hyperref[section:social]{social} and
%\hyperref[section:medical]{medical}
%pictograms.
It is a redistribution of the free (as in beer) \emph{Font Awesome} otf font with specific bindings for \hologo{(La)TeX}.

\section{Requirements}
The \textsf{fontawsome} package requires the \textsf{fontspec} package and either the \hologo{Xe}\hologo{(La)TeX} or Lua\hologo{(La)TeX} engine to load the included otf font.

\section{Usage}
\DescribeMacro{\faicon}
Once the \textsf{fontawesome} package loaded, icons can be accessed through the general \cs{faicon}, which takes as mandatory argument the \meta{name} of the desired icon, or through a direct command specific to each icon. The full list of icon designs, names and direct commands are showcased next.

\newenvironment{showcase}%
  {%
% \begin{longtable}{ccp{3cm}p{3.5cm}p{1cm}}% debug: shows icons with both generic and specific commands
   \begin{longtable}{cp{5cm}p{5cm}p{3cm}}
   \cmidrule[\heavyrulewidth]{1-3}% \toprule
% \bfseries Icon& \bfseries Icon& \bfseries Name& \bfseries Direct command& \\% debug
   \bfseries Icon& \bfseries Name& \bfseries Direct command& \\
   \cmidrule{1-3}\endhead}
  {\cmidrule[\heavyrulewidth]{1-3}% \bottomrule
   \end{longtable}}
\NewDocumentCommand{\showcaseicon}{mmg}{%
% \faicon{#1}& \csname#2\endcsname& \itshape #1& \ttfamily \textbackslash #2\index{\ttfamily \textbackslash #2}& \IfNoValueTF{#3}{}{\tag{#3}}\\}% debug
  \faicon{#1}& \itshape #1& \ttfamily \textbackslash #2\index{\ttfamily \textbackslash #2}& \IfNoValueTF{#3}{}{\tag{#3}}\\}
\newcommand{\tag}[1]{{%
  \small\sffamily%
  \ifthenelse{\equal{#1}{\currentversion}}{%
    \tikz[baseline={(TAG.base)}]{
      \node[white, fill=myblue, rounded corners=3pt, inner sep=1.5pt] (TAG) {New in \currentversion\vphantom{Ay!}};
    }}{
	\ifthenelse{\equal{#1}{gone}}{%
    		\tikz[baseline={(TAG.base)}]{
      		\node[black!50, fill=black!25, rounded corners=3pt, inner sep=1.5pt] (TAG) {gone (?)\vphantom{Ay!}};
    }}	{
  		\ifthenelse{\equal{#1}{alias}}{%
    			\textcolor{black!50}{(alias)}}{
    				\tikz[baseline={(TAG.base)}]{
      				\node[white, fill=black!65, rounded corners=3pt, inner sep=1.5pt] (TAG) {Since #1\vphantom{Ay!}};
				}}%
		}%
	}%
  }}

\subsection{Web Application Icons}
\begin{showcase}
\showcaseicon{glass}{faGlass}{1.0}% unicode:f000 / created:1.0
\showcaseicon{music}{faMusic}{1.0}% unicode:f001 / created:1.0
\showcaseicon{search}{faSearch}{1.0}% unicode:f002 / created:1.0
\showcaseicon{envelope-o}{faEnvelopeO}{1.0}% unicode:f003 / created:1.0
\showcaseicon{heart}{faHeart}{1.0}% unicode:f004 / created:1.0
\showcaseicon{star}{faStar}{1.0}% unicode:f005 / created:1.0
\showcaseicon{star-o}{faStarO}{1.0}% unicode:f006 / created:1.0
\showcaseicon{user}{faUser}{1.0}% unicode:f007 / created:1.0
\showcaseicon{film}{faFilm}{1.0}% unicode:f008 / created:1.0
\showcaseicon{check}{faCheck}{1.0}% unicode:f00c / created:1.0
\showcaseicon{times}{faTimes}{1.0}% unicode:f00d / created:1.0
\showcaseicon{search-plus}{faSearchPlus}{1.0}% unicode:f00e / created:1.0
\showcaseicon{search-minus}{faSearchMinus}{1.0}% unicode:f010 / created:1.0
\showcaseicon{power-off}{faPowerOff}{1.0}% unicode:f011 / created:1.0
\showcaseicon{signal}{faSignal}{1.0}% unicode:f012 / created:1.0
\showcaseicon{cog}{faCog}{1.0}% unicode:f013 / created:1.0
\showcaseicon{gear}{faGear}{alias}
\showcaseicon{trash-o}{faTrashO}{1.0}% unicode:f014 / created:1.0
\showcaseicon{home}{faHome}{1.0}% unicode:f015 / created:1.0
\showcaseicon{clock-o}{faClockO}{1.0}% unicode:f017 / created:1.0
\showcaseicon{road}{faRoad}{1.0}% unicode:f018 / created:1.0
\showcaseicon{download}{faDownload}{1.0}% unicode:f019 / created:1.0
\showcaseicon{inbox}{faInbox}{1.0}% unicode:f01c / created:1.0
\showcaseicon{refresh}{faRefresh}{1.0}% unicode:f021 / created:1.0
\showcaseicon{lock}{faLock}{1.0}% unicode:f023 / created:1.0
\showcaseicon{flag}{faFlag}{1.0}% unicode:f024 / created:1.0
\showcaseicon{headphones}{faHeadphones}{1.0}% unicode:f025 / created:1.0
\showcaseicon{volume-off}{faVolumeOff}{1.0}% unicode:f026 / created:1.0
\showcaseicon{volume-down}{faVolumeDown}{1.0}% unicode:f027 / created:1.0
\showcaseicon{volume-up}{faVolumeUp}{1.0}% unicode:f028 / created:1.0
\showcaseicon{qrcode}{faQrcode}{1.0}% unicode:f029 / created:1.0
\showcaseicon{barcode}{faBarcode}{1.0}% unicode:f02a / created:1.0
\showcaseicon{tag}{faTag}{1.0}% unicode:f02b / created:1.0
\showcaseicon{tags}{faTags}{1.0}% unicode:f02c / created:1.0
\showcaseicon{book}{faBook}{1.0}% unicode:f02d / created:1.0
\showcaseicon{bookmark}{faBookmark}{1.0}% unicode:f02e / created:1.0
\showcaseicon{print}{faPrint}{1.0}% unicode:f02f / created:1.0
\showcaseicon{camera}{faCamera}{1.0}% unicode:f030 / created:1.0
\showcaseicon{video-camera}{faVideoCamera}{1.0}% unicode:f03d / created:1.0
\showcaseicon{picture-o}{faPictureO}{1.0}% unicode:f03e / created:1.0
\showcaseicon{photo}{faPhoto}{alias}
\showcaseicon{image}{faImage}{alias}
\showcaseicon{pencil}{faPencil}{1.0}% unicode:f040 / created:1.0
\showcaseicon{map-marker}{faMapMarker}{1.0}% unicode:f041 / created:1.0
\showcaseicon{adjust}{faAdjust}{1.0}% unicode:f042 / created:1.0
\showcaseicon{tint}{faTint}{1.0}% unicode:f043 / created:1.0
\showcaseicon{pencil-square-o}{faPencilSquareO}{1.0}% unicode:f044 / created:1.0
\showcaseicon{edit}{faEdit}{alias}
\showcaseicon{share-square-o}{faShareSquareO}{1.0}% unicode:f045 / created:1.0
\showcaseicon{check-square-o}{faCheckSquareO}{1.0}% unicode:f046 / created:1.0
\showcaseicon{arrows}{faArrows}{1.0}% unicode:f047 / created:1.0
\showcaseicon{plus-circle}{faPlusCircle}{1.0}% unicode:f055 / created:1.0
\showcaseicon{minus-circle}{faMinusCircle}{1.0}% unicode:f056 / created:1.0
\showcaseicon{times-circle}{faTimesCircle}{1.0}% unicode:f057 / created:1.0
\showcaseicon{check-circle}{faCheckCircle}{1.0}% unicode:f058 / created:1.0
\showcaseicon{question-circle}{faQuestionCircle}{1.0}% unicode:f059 / created:1.0
\showcaseicon{info-circle}{faInfoCircle}{1.0}% unicode:f05a / created:1.0
\showcaseicon{crosshairs}{faCrosshairs}{1.0}% unicode:f05b / created:1.0
\showcaseicon{times-circle-o}{faTimesCircleO}{1.0}% unicode:f05c / created:1.0
\showcaseicon{check-circle-o}{faCheckCircleO}{1.0}% unicode:f05d / created:1.0
\showcaseicon{ban}{faBan}{1.0}% unicode:f05e / created:1.0
\showcaseicon{share}{faShare}{1.0}% unicode:f064 / created:1.0
\showcaseicon{mail-forward}{faMailForward}{alias}
\showcaseicon{plus}{faPlus}{1.0}% unicode:f067 / created:1.0
\showcaseicon{minus}{faMinus}{1.0}% unicode:f068 / created:1.0
\showcaseicon{asterisk}{faAsterisk}{1.0}% unicode:f069 / created:1.0
\showcaseicon{exclamation-circle}{faExclamationCircle}{1.0}% unicode:f06a / created:1.0
\showcaseicon{gift}{faGift}{1.0}% unicode:f06b / created:1.0
\showcaseicon{leaf}{faLeaf}{1.0}% unicode:f06c / created:1.0
\showcaseicon{fire}{faFire}{1.0}% unicode:f06d / created:1.0
\showcaseicon{eye}{faEye}{1.0}% unicode:f06e / created:1.0
\showcaseicon{eye-slash}{faEyeSlash}{1.0}% unicode:f070 / created:1.0
\showcaseicon{exclamation-triangle}{faExclamationTriangle}{1.0}% unicode:f071 / created:1.0
\showcaseicon{warning}{faWarning}{alias}
\showcaseicon{plane}{faPlane}{1.0}% unicode:f072 / created:1.0
\showcaseicon{calendar}{faCalendar}{1.0}% unicode:f073 / created:1.0
\showcaseicon{random}{faRandom}{1.0}% unicode:f074 / created:1.0
\showcaseicon{comment}{faComment}{1.0}% unicode:f075 / created:1.0
\showcaseicon{magnet}{faMagnet}{1.0}% unicode:f076 / created:1.0
\showcaseicon{retweet}{faRetweet}{1.0}% unicode:f079 / created:1.0
\showcaseicon{shopping-cart}{faShoppingCart}{1.0}% unicode:f07a / created:1.0
\showcaseicon{folder}{faFolder}{1.0}% unicode:f07b / created:1.0
\showcaseicon{folder-open}{faFolderOpen}{1.0}% unicode:f07c / created:1.0
\showcaseicon{arrows-v}{faArrowsV}{1.0}% unicode:f07d / created:1.0
\showcaseicon{arrows-h}{faArrowsH}{1.0}% unicode:f07e / created:1.0
\showcaseicon{bar-chart-o}{faBarChartO}{1.0}% unicode:f080 / created:1.0
\showcaseicon{camera-retro}{faCameraRetro}{1.0}% unicode:f083 / created:1.0
\showcaseicon{key}{faKey}{1.0}% unicode:f084 / created:1.0
\showcaseicon{cogs}{faCogs}{1.0}% unicode:f085 / created:1.0
\showcaseicon{gears}{faGears}{alias}
\showcaseicon{comments}{faComments}{1.0}% unicode:f086 / created:1.0
\showcaseicon{thumbs-o-up}{faThumbsOUp}{1.0}% unicode:f087 / created:1.0
\showcaseicon{thumbs-o-down}{faThumbsODown}{1.0}% unicode:f088 / created:1.0
\showcaseicon{star-half}{faStarHalf}{1.0}% unicode:f089 / created:1.0
\showcaseicon{heart-o}{faHeartO}{1.0}% unicode:f08a / created:1.0
\showcaseicon{sign-out}{faSignOut}{1.0}% unicode:f08b / created:1.0
\showcaseicon{thumb-tack}{faThumbTack}{1.0}% unicode:f08d / created:1.0
\showcaseicon{external-link}{faExternalLink}{1.0}% unicode:f08e / created:1.0
\showcaseicon{sign-in}{faSignIn}{1.0}% unicode:f090 / created:1.0
\showcaseicon{trophy}{faTrophy}{1.0}% unicode:f091 / created:1.0
\showcaseicon{upload}{faUpload}{1.0}% unicode:f093 / created:1.0
\showcaseicon{lemon-o}{faLemonO}{1.0}% unicode:f094 / created:1.0
\showcaseicon{phone}{faPhone}{2.0}% unicode:f095 / created:2.0
\showcaseicon{square-o}{faSquareO}{2.0}% unicode:f096 / created:2.0
\showcaseicon{bookmark-o}{faBookmarkO}{2.0}% unicode:f097 / created:2.0
\showcaseicon{phone-square}{faPhoneSquare}{2.0}% unicode:f098 / created:2.0
\showcaseicon{unlock}{faUnlock}{2.0}% unicode:f09c / created:2.0
\showcaseicon{credit-card}{faCreditCard}{2.0}% unicode:f09d / created:2.0
\showcaseicon{rss}{faRss}{2.0}% unicode:f09e / created:2.0
\showcaseicon{hdd-o}{faHddO}{2.0}% unicode:f0a0 / created:2.0
\showcaseicon{bullhorn}{faBullhorn}{2.0}% unicode:f0a1 / created:2.0
\showcaseicon{bell}{faBell}{2.0}% unicode:f0f3 / created:2.0
\showcaseicon{certificate}{faCertificate}{2.0}% unicode:f0a3 / created:2.0
\showcaseicon{globe}{faGlobe}{2.0}% unicode:f0ac / created:2.0
\showcaseicon{wrench}{faWrench}{2.0}% unicode:f0ad / created:2.0
\showcaseicon{tasks}{faTasks}{2.0}% unicode:f0ae / created:2.0
\showcaseicon{filter}{faFilter}{2.0}% unicode:f0b0 / created:2.0
\showcaseicon{briefcase}{faBriefcase}{2.0}% unicode:f0b1 / created:2.0
\showcaseicon{users}{faUsers}{2.0}% unicode:f0c0 / created:2.0
\showcaseicon{group}{faGroup}{alias}
\showcaseicon{cloud}{faCloud}{2.0}% unicode:f0c2 / created:2.0
\showcaseicon{flask}{faFlask}{2.0}% unicode:f0c3 / created:2.0
\showcaseicon{square}{faSquare}{2.0}% unicode:f0c8 / created:2.0
\showcaseicon{bars}{faBars}{2.0}% unicode:f0c9 / created:2.0
\showcaseicon{navicon}{faNavicon}{alias}
\showcaseicon{reorder}{faReorder}{alias}
\showcaseicon{magic}{faMagic}{2.0}% unicode:f0d0 / created:2.0
\showcaseicon{truck}{faTruck}{2.0}% unicode:f0d1 / created:2.0
\showcaseicon{money}{faMoney}{2.0}% unicode:f0d6 / created:2.0
\showcaseicon{sort}{faSort}{2.0}% unicode:f0dc / created:2.0
\showcaseicon{unsorted}{faUnsorted}{alias}
\showcaseicon{sort-desc}{faSortDesc}{2.0}% unicode:f0dd / created:2.0
\showcaseicon{sort-down}{faSortDown}{alias}
\showcaseicon{sort-asc}{faSortAsc}{2.0}% unicode:f0de / created:2.0
\showcaseicon{sort-up}{faSortUp}{alias}
\showcaseicon{envelope}{faEnvelope}{2.0}% unicode:f0e0 / created:2.0
\showcaseicon{gavel}{faGavel}{2.0}% unicode:f0e3 / created:2.0
\showcaseicon{legal}{faLegal}{alias}
\showcaseicon{tachometer}{faTachometer}{2.0}% unicode:f0e4 / created:2.0
\showcaseicon{dashboard}{faDashboard}{alias}
\showcaseicon{comment-o}{faCommentO}{2.0}% unicode:f0e5 / created:2.0
\showcaseicon{comments-o}{faCommentsO}{2.0}% unicode:f0e6 / created:2.0
\showcaseicon{bolt}{faBolt}{2.0}% unicode:f0e7 / created:2.0
\showcaseicon{flash}{faFlash}{alias}
\showcaseicon{sitemap}{faSitemap}{2.0}% unicode:f0e8 / created:2.0
\showcaseicon{umbrella}{faUmbrella}{2.0}% unicode:f0e9 / created:2.0
\showcaseicon{lightbulb-o}{faLightbulbO}{3.0}% unicode:f0eb / created:3.0
\showcaseicon{exchange}{faExchange}{3.0}% unicode:f0ec / created:3.0
\showcaseicon{cloud-download}{faCloudDownload}{3.0}% unicode:f0ed / created:3.0
\showcaseicon{cloud-upload}{faCloudUpload}{3.0}% unicode:f0ee / created:3.0
\showcaseicon{suitcase}{faSuitcase}{3.0}% unicode:f0f2 / created:3.0
\showcaseicon{bell-o}{faBellO}{3.0}% unicode:f0a2 / created:3.0
\showcaseicon{coffee}{faCoffee}{3.0}% unicode:f0f4 / created:3.0
\showcaseicon{cutlery}{faCutlery}{3.0}% unicode:f0f5 / created:3.0
\showcaseicon{building-o}{faBuildingO}{3.0}% unicode:f0f7 / created:3.0
\showcaseicon{fighter-jet}{faFighterJet}{3.0}% unicode:f0fb / created:3.0
\showcaseicon{beer}{faBeer}{3.0}% unicode:f0fc / created:3.0
\showcaseicon{plus-square}{faPlusSquare}{3.0}% unicode:f0fe / created:3.0
\showcaseicon{desktop}{faDesktop}{3.0}% unicode:f108 / created:3.0
\showcaseicon{laptop}{faLaptop}{3.0}% unicode:f109 / created:3.0
\showcaseicon{tablet}{faTablet}{3.0}% unicode:f10a / created:3.0
\showcaseicon{mobile}{faMobile}{3.0}% unicode:f10b / created:3.0
\showcaseicon{mobile-phone}{faMobilePhone}{alias}
\showcaseicon{circle-o}{faCircleO}{3.0}% unicode:f10c / created:3.0
\showcaseicon{quote-left}{faQuoteLeft}{3.0}% unicode:f10d / created:3.0
\showcaseicon{quote-right}{faQuoteRight}{3.0}% unicode:f10e / created:3.0
\showcaseicon{spinner}{faSpinner}{3.0}% unicode:f110 / created:3.0
\showcaseicon{circle}{faCircle}{3.0}% unicode:f111 / created:3.0
\showcaseicon{reply}{faReply}{3.0}% unicode:f112 / created:3.0
\showcaseicon{mail-reply}{faMailReply}{alias}
\showcaseicon{folder-o}{faFolderO}{3.0}% unicode:f114 / created:3.0
\showcaseicon{folder-open-o}{faFolderOpenO}{3.0}% unicode:f115 / created:3.0
\showcaseicon{smile-o}{faSmileO}{3.1}% unicode:f118 / created:3.1
\showcaseicon{frown-o}{faFrownO}{3.1}% unicode:f119 / created:3.1
\showcaseicon{meh-o}{faMehO}{3.1}% unicode:f11a / created:3.1
\showcaseicon{gamepad}{faGamepad}{3.1}% unicode:f11b / created:3.1
\showcaseicon{keyboard-o}{faKeyboardO}{3.1}% unicode:f11c / created:3.1
\showcaseicon{flag-o}{faFlagO}{3.1}% unicode:f11d / created:3.1
\showcaseicon{flag-checkered}{faFlagCheckered}{3.1}% unicode:f11e / created:3.1
\showcaseicon{terminal}{faTerminal}{3.1}% unicode:f120 / created:3.1
\showcaseicon{code}{faCode}{3.1}% unicode:f121 / created:3.1
\showcaseicon{reply-all}{faReplyAll}{3.1}% unicode:f122 / created:3.1
\showcaseicon{mail-reply-all}{faMailReplyAll}{alias}
\showcaseicon{star-half-o}{faStarHalfO}{3.1}% unicode:f123 / created:3.1
\showcaseicon{star-half-empty}{faStarHalfEmpty}{alias}
\showcaseicon{star-half-full}{faStarHalfFull}{alias}
\showcaseicon{location-arrow}{faLocationArrow}{3.1}% unicode:f124 / created:3.1
\showcaseicon{crop}{faCrop}{3.1}% unicode:f125 / created:3.1
\showcaseicon{code-fork}{faCodeFork}{3.1}% unicode:f126 / created:3.1
\showcaseicon{question}{faQuestion}{3.1}% unicode:f128 / created:3.1
\showcaseicon{info}{faInfo}{3.1}% unicode:f129 / created:3.1
\showcaseicon{exclamation}{faExclamation}{3.1}% unicode:f12a / created:3.1
\showcaseicon{eraser}{faEraser}{3.1}% unicode:f12d / created:3.1
\showcaseicon{puzzle-piece}{faPuzzlePiece}{3.1}% unicode:f12e / created:3.1
\showcaseicon{microphone}{faMicrophone}{3.1}% unicode:f130 / created:3.1
\showcaseicon{microphone-slash}{faMicrophoneSlash}{3.1}% unicode:f131 / created:3.1
\showcaseicon{shield}{faShield}{3.1}% unicode:f132 / created:3.1
\showcaseicon{calendar-o}{faCalendarO}{3.1}% unicode:f133 / created:3.1
\showcaseicon{fire-extinguisher}{faFireExtinguisher}{3.1}% unicode:f134 / created:3.1
\showcaseicon{rocket}{faRocket}{3.1}% unicode:f135 / created:3.1
\showcaseicon{anchor}{faAnchor}{3.1}% unicode:f13d / created:3.1
\showcaseicon{unlock-alt}{faUnlockAlt}{3.1}% unicode:f13e / created:3.1
\showcaseicon{bullseye}{faBullseye}{3.1}% unicode:f140 / created:3.1
\showcaseicon{ellipsis-h}{faEllipsisH}{3.1}% unicode:f141 / created:3.1
\showcaseicon{ellipsis-v}{faEllipsisV}{3.1}% unicode:f142 / created:3.1
\showcaseicon{rss-square}{faRssSquare}{3.1}% unicode:f143 / created:3.1
\showcaseicon{ticket}{faTicket}{3.1}% unicode:f145 / created:3.1
\showcaseicon{minus-square}{faMinusSquare}{3.1}% unicode:f146 / created:3.1
\showcaseicon{minus-square-o}{faMinusSquareO}{3.1}% unicode:f147 / created:3.1
\showcaseicon{level-up}{faLevelUp}{3.1}% unicode:f148 / created:3.1
\showcaseicon{level-down}{faLevelDown}{3.1}% unicode:f149 / created:3.1
\showcaseicon{check-square}{faCheckSquare}{3.1}% unicode:f14a / created:3.1
\showcaseicon{pencil-square}{faPencilSquare}{3.1}% unicode:f14b / created:3.1
\showcaseicon{external-link-square}{faExternalLinkSquare}{3.1}% unicode:f14c / created:3.1
\showcaseicon{share-square}{faShareSquare}{3.1}% unicode:f14d / created:3.1
\showcaseicon{compass}{faCompass}{3.2}% unicode:f14e / created:3.2
\showcaseicon{caret-square-o-down}{faCaretSquareODown}{3.2}% unicode:f150 / created:3.2
\showcaseicon{toggle-down}{faToggleDown}{alias}
\showcaseicon{caret-square-o-up}{faCaretSquareOUp}{3.2}% unicode:f151 / created:3.2
\showcaseicon{toggle-up}{faToggleUp}{alias}
\showcaseicon{caret-square-o-right}{faCaretSquareORight}{3.2}% unicode:f152 / created:3.2
\showcaseicon{toggle-right}{faToggleRight}{alias}
\showcaseicon{sort-alpha-asc}{faSortAlphaAsc}{3.2}% unicode:f15d / created:3.2
\showcaseicon{sort-alpha-desc}{faSortAlphaDesc}{3.2}% unicode:f15e / created:3.2
\showcaseicon{sort-amount-asc}{faSortAmountAsc}{3.2}% unicode:f160 / created:3.2
\showcaseicon{sort-amount-desc}{faSortAmountDesc}{3.2}% unicode:f161 / created:3.2
\showcaseicon{sort-numeric-asc}{faSortNumericAsc}{3.2}% unicode:f162 / created:3.2
\showcaseicon{sort-numeric-desc}{faSortNumericDesc}{3.2}% unicode:f163 / created:3.2
\showcaseicon{thumbs-up}{faThumbsUp}{3.2}% unicode:f164 / created:3.2
\showcaseicon{thumbs-down}{faThumbsDown}{3.2}% unicode:f165 / created:3.2
\showcaseicon{female}{faFemale}{3.2}% unicode:f182 / created:3.2
\showcaseicon{male}{faMale}{3.2}% unicode:f183 / created:3.2
\showcaseicon{sun-o}{faSunO}{3.2}% unicode:f185 / created:3.2
\showcaseicon{moon-o}{faMoonO}{3.2}% unicode:f186 / created:3.2
\showcaseicon{archive}{faArchive}{3.2}% unicode:f187 / created:3.2
\showcaseicon{bug}{faBug}{3.2}% unicode:f188 / created:3.2
\showcaseicon{caret-square-o-left}{faCaretSquareOLeft}{4.0}% unicode:f191 / created:4.0
\showcaseicon{toggle-left}{faToggleLeft}{alias}
\showcaseicon{dot-circle-o}{faDotCircleO}{4.0}% unicode:f192 / created:4.0
\showcaseicon{wheelchair}{faWheelchair}{4.0}% unicode:f193 / created:4.0
\showcaseicon{plus-square-o}{faPlusSquareO}{4.0}% unicode:f196 / created:4.0
\showcaseicon{space-shuttle}{faSpaceShuttle}{4.1}% unicode:f197 / created:4.1
\showcaseicon{envelope-square}{faEnvelopeSquare}{4.1}% unicode:f199 / created:4.1
\showcaseicon{university}{faUniversity}{4.1}% unicode:f19c / created:4.1
\showcaseicon{institution}{faInstitution}{alias}
\showcaseicon{bank}{faBank}{alias}
\showcaseicon{graduation-cap}{faGraduationCap}{4.1}% unicode:f19d / created:4.1
\showcaseicon{mortar-board}{faMortarBoard}{alias}
\showcaseicon{language}{faLanguage}{4.1}% unicode:f1ab / created:4.1
\showcaseicon{fax}{faFax}{4.1}% unicode:f1ac / created:4.1
\showcaseicon{building}{faBuilding}{4.1}% unicode:f1ad / created:4.1
\showcaseicon{child}{faChild}{4.1}% unicode:f1ae / created:4.1
\showcaseicon{paw}{faPaw}{4.1}% unicode:f1b0 / created:4.1
\showcaseicon{spoon}{faSpoon}{4.1}% unicode:f1b1 / created:4.1
\showcaseicon{cube}{faCube}{4.1}% unicode:f1b2 / created:4.1
\showcaseicon{cubes}{faCubes}{4.1}% unicode:f1b3 / created:4.1
\showcaseicon{recycle}{faRecycle}{4.1}% unicode:f1b8 / created:4.1
\showcaseicon{car}{faCar}{4.1}% unicode:f1b9 / created:4.1
\showcaseicon{automobile}{faAutomobile}{alias}
\showcaseicon{taxi}{faTaxi}{4.1}% unicode:f1ba / created:4.1
\showcaseicon{cab}{faCab}{alias}
\showcaseicon{tree}{faTree}{4.1}% unicode:f1bb / created:4.1
\showcaseicon{database}{faDatabase}{4.1}% unicode:f1c0 / created:4.1
\showcaseicon{file-pdf-o}{faFilePdfO}{4.1}% unicode:f1c1 / created:4.1
\showcaseicon{file-word-o}{faFileWordO}{4.1}% unicode:f1c2 / created:4.1
\showcaseicon{file-excel-o}{faFileExcelO}{4.1}% unicode:f1c3 / created:4.1
\showcaseicon{file-powerpoint-o}{faFilePowerpointO}{4.1}% unicode:f1c4 / created:4.1
\showcaseicon{file-image-o}{faFileImageO}{4.1}% unicode:f1c5 / created:4.1
\showcaseicon{file-photo-o}{faFilePhotoO}{alias}
\showcaseicon{file-picture-o}{faFilePictureO}{alias}
\showcaseicon{file-archive-o}{faFileArchiveO}{4.1}% unicode:f1c6 / created:4.1
\showcaseicon{file-zip-o}{faFileZipO}{alias}
\showcaseicon{file-audio-o}{faFileAudioO}{4.1}% unicode:f1c7 / created:4.1
\showcaseicon{file-sound-o}{faFileSoundO}{alias}
\showcaseicon{file-video-o}{faFileVideoO}{4.1}% unicode:f1c8 / created:4.1
\showcaseicon{file-movie-o}{faFileMovieO}{alias}
\showcaseicon{file-code-o}{faFileCodeO}{4.1}% unicode:f1c9 / created:4.1
\showcaseicon{life-ring}{faLifeRing}{4.1}% unicode:f1cd / created:4.1
\showcaseicon{life-bouy}{faLifeBouy}{alias}
\showcaseicon{life-saver}{faLifeSaver}{alias}
\showcaseicon{support}{faSupport}{alias}
\showcaseicon{circle-o-notch}{faCircleONotch}{4.1}% unicode:f1ce / created:4.1
\showcaseicon{paper-plane}{faPaperPlane}{4.1}% unicode:f1d8 / created:4.1
\showcaseicon{send}{faSend}{alias}
\showcaseicon{paper-plane-o}{faPaperPlaneO}{4.1}% unicode:f1d9 / created:4.1
\showcaseicon{send-o}{faSendO}{alias}
\showcaseicon{history}{faHistory}{4.1}% unicode:f1da / created:4.1
\showcaseicon{circle-thin}{faCircleThin}{4.1}% unicode:f1db / created:4.1
\showcaseicon{sliders}{faSliders}{4.1}% unicode:f1de / created:4.1
\showcaseicon{share-alt}{faShareAlt}{4.1}% unicode:f1e0 / created:4.1
\showcaseicon{share-alt-square}{faShareAltSquare}{4.1}% unicode:f1e1 / created:4.1
\showcaseicon{bomb}{faBomb}{4.1}% unicode:f1e2 / created:4.1
\end{showcase}
\subsection{Text Editor Icons}
\begin{showcase}
\showcaseicon{th-large}{faThLarge}{1.0}% unicode:f009 / created:1.0
\showcaseicon{th}{faTh}{1.0}% unicode:f00a / created:1.0
\showcaseicon{th-list}{faThList}{1.0}% unicode:f00b / created:1.0
\showcaseicon{file-o}{faFileO}{1.0}% unicode:f016 / created:1.0
\showcaseicon{repeat}{faRepeat}{1.0}% unicode:f01e / created:1.0
\showcaseicon{rotate-right}{faRotateRight}{alias}
\showcaseicon{list-alt}{faListAlt}{1.0}% unicode:f022 / created:1.0
\showcaseicon{font}{faFont}{1.0}% unicode:f031 / created:1.0
\showcaseicon{bold}{faBold}{1.0}% unicode:f032 / created:1.0
\showcaseicon{italic}{faItalic}{1.0}% unicode:f033 / created:1.0
\showcaseicon{text-height}{faTextHeight}{1.0}% unicode:f034 / created:1.0
\showcaseicon{text-width}{faTextWidth}{1.0}% unicode:f035 / created:1.0
\showcaseicon{align-left}{faAlignLeft}{1.0}% unicode:f036 / created:1.0
\showcaseicon{align-center}{faAlignCenter}{1.0}% unicode:f037 / created:1.0
\showcaseicon{align-right}{faAlignRight}{1.0}% unicode:f038 / created:1.0
\showcaseicon{align-justify}{faAlignJustify}{1.0}% unicode:f039 / created:1.0
\showcaseicon{list}{faList}{1.0}% unicode:f03a / created:1.0
\showcaseicon{outdent}{faOutdent}{1.0}% unicode:f03b / created:1.0
\showcaseicon{dedent}{faDedent}{alias}
\showcaseicon{indent}{faIndent}{1.0}% unicode:f03c / created:1.0
\showcaseicon{link}{faLink}{2.0}% unicode:f0c1 / created:2.0
\showcaseicon{chain}{faChain}{alias}
\showcaseicon{scissors}{faScissors}{2.0}% unicode:f0c4 / created:2.0
\showcaseicon{cut}{faCut}{alias}
\showcaseicon{files-o}{faFilesO}{2.0}% unicode:f0c5 / created:2.0
\showcaseicon{copy}{faCopy}{alias}
\showcaseicon{paperclip}{faPaperclip}{2.0}% unicode:f0c6 / created:2.0
\showcaseicon{floppy-o}{faFloppyO}{2.0}% unicode:f0c7 / created:2.0
\showcaseicon{save}{faSave}{alias}
\showcaseicon{list-ul}{faListUl}{2.0}% unicode:f0ca / created:2.0
\showcaseicon{list-ol}{faListOl}{2.0}% unicode:f0cb / created:2.0
\showcaseicon{strikethrough}{faStrikethrough}{2.0}% unicode:f0cc / created:2.0
\showcaseicon{underline}{faUnderline}{2.0}% unicode:f0cd / created:2.0
\showcaseicon{table}{faTable}{2.0}% unicode:f0ce / created:2.0
\showcaseicon{columns}{faColumns}{2.0}% unicode:f0db / created:2.0
\showcaseicon{undo}{faUndo}{2.0}% unicode:f0e2 / created:2.0
\showcaseicon{rotate-left}{faRotateLeft}{alias}
\showcaseicon{clipboard}{faClipboard}{2.0}% unicode:f0ea / created:2.0
\showcaseicon{paste}{faPaste}{alias}
\showcaseicon{file-text-o}{faFileTextO}{3.0}% unicode:f0f6 / created:3.0
\showcaseicon{chain-broken}{faChainBroken}{3.1}% unicode:f127 / created:3.1
\showcaseicon{unlink}{faUnlink}{alias}
\showcaseicon{superscript}{faSuperscript}{3.1}% unicode:f12b / created:3.1
\showcaseicon{subscript}{faSubscript}{3.1}% unicode:f12c / created:3.1
\showcaseicon{eraser}{faEraser}{3.1}% unicode:f12d / created:3.1
\showcaseicon{file}{faFile}{3.2}% unicode:f15b / created:3.2
\showcaseicon{file-text}{faFileText}{3.2}% unicode:f15c / created:3.2
\showcaseicon{header}{faHeader}{4.1}% unicode:f1dc / created:4.1
\showcaseicon{paragraph}{faParagraph}{4.1}% unicode:f1dd / created:4.1
\end{showcase}
\subsection{Spinner Icons}
\begin{showcase}
\showcaseicon{cog}{faCog}{1.0}% unicode:f013 / created:1.0
\showcaseicon{gear}{faGear}{alias}
\showcaseicon{refresh}{faRefresh}{1.0}% unicode:f021 / created:1.0
\showcaseicon{spinner}{faSpinner}{3.0}% unicode:f110 / created:3.0
\showcaseicon{circle-o-notch}{faCircleONotch}{4.1}% unicode:f1ce / created:4.1
\end{showcase}
\subsection{File Type Icons}
\begin{showcase}
\showcaseicon{file-o}{faFileO}{1.0}% unicode:f016 / created:1.0
\showcaseicon{file-text-o}{faFileTextO}{3.0}% unicode:f0f6 / created:3.0
\showcaseicon{file}{faFile}{3.2}% unicode:f15b / created:3.2
\showcaseicon{file-text}{faFileText}{3.2}% unicode:f15c / created:3.2
\showcaseicon{file-pdf-o}{faFilePdfO}{4.1}% unicode:f1c1 / created:4.1
\showcaseicon{file-word-o}{faFileWordO}{4.1}% unicode:f1c2 / created:4.1
\showcaseicon{file-excel-o}{faFileExcelO}{4.1}% unicode:f1c3 / created:4.1
\showcaseicon{file-powerpoint-o}{faFilePowerpointO}{4.1}% unicode:f1c4 / created:4.1
\showcaseicon{file-image-o}{faFileImageO}{4.1}% unicode:f1c5 / created:4.1
\showcaseicon{file-photo-o}{faFilePhotoO}{alias}
\showcaseicon{file-picture-o}{faFilePictureO}{alias}
\showcaseicon{file-archive-o}{faFileArchiveO}{4.1}% unicode:f1c6 / created:4.1
\showcaseicon{file-zip-o}{faFileZipO}{alias}
\showcaseicon{file-audio-o}{faFileAudioO}{4.1}% unicode:f1c7 / created:4.1
\showcaseicon{file-sound-o}{faFileSoundO}{alias}
\showcaseicon{file-video-o}{faFileVideoO}{4.1}% unicode:f1c8 / created:4.1
\showcaseicon{file-movie-o}{faFileMovieO}{alias}
\showcaseicon{file-code-o}{faFileCodeO}{4.1}% unicode:f1c9 / created:4.1
\end{showcase}
\subsection{Directional Icons}
\begin{showcase}
\showcaseicon{arrow-circle-o-down}{faArrowCircleODown}{1.0}% unicode:f01a / created:1.0
\showcaseicon{arrow-circle-o-up}{faArrowCircleOUp}{1.0}% unicode:f01b / created:1.0
\showcaseicon{arrows}{faArrows}{1.0}% unicode:f047 / created:1.0
\showcaseicon{chevron-left}{faChevronLeft}{1.0}% unicode:f053 / created:1.0
\showcaseicon{chevron-right}{faChevronRight}{1.0}% unicode:f054 / created:1.0
\showcaseicon{arrow-left}{faArrowLeft}{1.0}% unicode:f060 / created:1.0
\showcaseicon{arrow-right}{faArrowRight}{1.0}% unicode:f061 / created:1.0
\showcaseicon{arrow-up}{faArrowUp}{1.0}% unicode:f062 / created:1.0
\showcaseicon{arrow-down}{faArrowDown}{1.0}% unicode:f063 / created:1.0
\showcaseicon{chevron-up}{faChevronUp}{1.0}% unicode:f077 / created:1.0
\showcaseicon{chevron-down}{faChevronDown}{1.0}% unicode:f078 / created:1.0
\showcaseicon{arrows-v}{faArrowsV}{1.0}% unicode:f07d / created:1.0
\showcaseicon{arrows-h}{faArrowsH}{1.0}% unicode:f07e / created:1.0
\showcaseicon{hand-o-right}{faHandORight}{2.0}% unicode:f0a4 / created:2.0
\showcaseicon{hand-o-left}{faHandOLeft}{2.0}% unicode:f0a5 / created:2.0
\showcaseicon{hand-o-up}{faHandOUp}{2.0}% unicode:f0a6 / created:2.0
\showcaseicon{hand-o-down}{faHandODown}{2.0}% unicode:f0a7 / created:2.0
\showcaseicon{arrow-circle-left}{faArrowCircleLeft}{2.0}% unicode:f0a8 / created:2.0
\showcaseicon{arrow-circle-right}{faArrowCircleRight}{2.0}% unicode:f0a9 / created:2.0
\showcaseicon{arrow-circle-up}{faArrowCircleUp}{2.0}% unicode:f0aa / created:2.0
\showcaseicon{arrow-circle-down}{faArrowCircleDown}{2.0}% unicode:f0ab / created:2.0
\showcaseicon{arrows-alt}{faArrowsAlt}{2.0}% unicode:f0b2 / created:2.0
\showcaseicon{caret-down}{faCaretDown}{2.0}% unicode:f0d7 / created:2.0
\showcaseicon{caret-up}{faCaretUp}{2.0}% unicode:f0d8 / created:2.0
\showcaseicon{caret-left}{faCaretLeft}{2.0}% unicode:f0d9 / created:2.0
\showcaseicon{caret-right}{faCaretRight}{2.0}% unicode:f0da / created:2.0
\showcaseicon{angle-double-left}{faAngleDoubleLeft}{3.0}% unicode:f100 / created:3.0
\showcaseicon{angle-double-right}{faAngleDoubleRight}{3.0}% unicode:f101 / created:3.0
\showcaseicon{angle-double-up}{faAngleDoubleUp}{3.0}% unicode:f102 / created:3.0
\showcaseicon{angle-double-down}{faAngleDoubleDown}{3.0}% unicode:f103 / created:3.0
\showcaseicon{angle-left}{faAngleLeft}{3.0}% unicode:f104 / created:3.0
\showcaseicon{angle-right}{faAngleRight}{3.0}% unicode:f105 / created:3.0
\showcaseicon{angle-up}{faAngleUp}{3.0}% unicode:f106 / created:3.0
\showcaseicon{angle-down}{faAngleDown}{3.0}% unicode:f107 / created:3.0
\showcaseicon{chevron-circle-left}{faChevronCircleLeft}{3.1}% unicode:f137 / created:3.1
\showcaseicon{chevron-circle-right}{faChevronCircleRight}{3.1}% unicode:f138 / created:3.1
\showcaseicon{chevron-circle-up}{faChevronCircleUp}{3.1}% unicode:f139 / created:3.1
\showcaseicon{chevron-circle-down}{faChevronCircleDown}{3.1}% unicode:f13a / created:3.1
\showcaseicon{caret-square-o-down}{faCaretSquareODown}{3.2}% unicode:f150 / created:3.2
\showcaseicon{toggle-down}{faToggleDown}{alias}
\showcaseicon{caret-square-o-up}{faCaretSquareOUp}{3.2}% unicode:f151 / created:3.2
\showcaseicon{toggle-up}{faToggleUp}{alias}
\showcaseicon{caret-square-o-right}{faCaretSquareORight}{3.2}% unicode:f152 / created:3.2
\showcaseicon{toggle-right}{faToggleRight}{alias}
\showcaseicon{long-arrow-down}{faLongArrowDown}{3.2}% unicode:f175 / created:3.2
\showcaseicon{long-arrow-up}{faLongArrowUp}{3.2}% unicode:f176 / created:3.2
\showcaseicon{long-arrow-left}{faLongArrowLeft}{3.2}% unicode:f177 / created:3.2
\showcaseicon{long-arrow-right}{faLongArrowRight}{3.2}% unicode:f178 / created:3.2
\showcaseicon{arrow-circle-o-right}{faArrowCircleORight}{4.0}% unicode:f18e / created:4.0
\showcaseicon{arrow-circle-o-left}{faArrowCircleOLeft}{4.0}% unicode:f190 / created:4.0
\showcaseicon{caret-square-o-left}{faCaretSquareOLeft}{4.0}% unicode:f191 / created:4.0
\showcaseicon{toggle-left}{faToggleLeft}{alias}
\end{showcase}
\subsection{Video Player Icons}
\begin{showcase}
\showcaseicon{play-circle-o}{faPlayCircleO}{1.0}% unicode:f01d / created:1.0
\showcaseicon{step-backward}{faStepBackward}{1.0}% unicode:f048 / created:1.0
\showcaseicon{fast-backward}{faFastBackward}{1.0}% unicode:f049 / created:1.0
\showcaseicon{backward}{faBackward}{1.0}% unicode:f04a / created:1.0
\showcaseicon{play}{faPlay}{1.0}% unicode:f04b / created:1.0
\showcaseicon{pause}{faPause}{1.0}% unicode:f04c / created:1.0
\showcaseicon{stop}{faStop}{1.0}% unicode:f04d / created:1.0
\showcaseicon{forward}{faForward}{1.0}% unicode:f04e / created:1.0
\showcaseicon{fast-forward}{faFastForward}{1.0}% unicode:f050 / created:1.0
\showcaseicon{step-forward}{faStepForward}{1.0}% unicode:f051 / created:1.0
\showcaseicon{eject}{faEject}{1.0}% unicode:f052 / created:1.0
\showcaseicon{expand}{faExpand}{1.0}% unicode:f065 / created:1.0
\showcaseicon{compress}{faCompress}{1.0}% unicode:f066 / created:1.0
\showcaseicon{arrows-alt}{faArrowsAlt}{2.0}% unicode:f0b2 / created:2.0
\showcaseicon{play-circle}{faPlayCircle}{3.1}% unicode:f144 / created:3.1
\showcaseicon{youtube-play}{faYoutubePlay}{3.2}% unicode:f16a / created:3.2
\end{showcase}
\subsection{Form Control Icons}
\begin{showcase}
\showcaseicon{check-square-o}{faCheckSquareO}{1.0}% unicode:f046 / created:1.0
\showcaseicon{square-o}{faSquareO}{2.0}% unicode:f096 / created:2.0
\showcaseicon{square}{faSquare}{2.0}% unicode:f0c8 / created:2.0
\showcaseicon{plus-square}{faPlusSquare}{3.0}% unicode:f0fe / created:3.0
\showcaseicon{circle-o}{faCircleO}{3.0}% unicode:f10c / created:3.0
\showcaseicon{circle}{faCircle}{3.0}% unicode:f111 / created:3.0
\showcaseicon{minus-square}{faMinusSquare}{3.1}% unicode:f146 / created:3.1
\showcaseicon{minus-square-o}{faMinusSquareO}{3.1}% unicode:f147 / created:3.1
\showcaseicon{check-square}{faCheckSquare}{3.1}% unicode:f14a / created:3.1
\showcaseicon{dot-circle-o}{faDotCircleO}{4.0}% unicode:f192 / created:4.0
\showcaseicon{plus-square-o}{faPlusSquareO}{4.0}% unicode:f196 / created:4.0
\end{showcase}
\subsection{Brand Icons}
\begin{showcase}
\showcaseicon{twitter-square}{faTwitterSquare}{1.0}% unicode:f081 / created:1.0
\showcaseicon{facebook-square}{faFacebookSquare}{1.0}% unicode:f082 / created:1.0
\showcaseicon{linkedin-square}{faLinkedinSquare}{1.0}% unicode:f08c / created:1.0
\showcaseicon{github-square}{faGithubSquare}{1.0}% unicode:f092 / created:1.0
\showcaseicon{twitter}{faTwitter}{2.0}% unicode:f099 / created:2.0
\showcaseicon{facebook}{faFacebook}{2.0}% unicode:f09a / created:2.0
\showcaseicon{github}{faGithub}{2.0}% unicode:f09b / created:2.0
\showcaseicon{pinterest}{faPinterest}{2.0}% unicode:f0d2 / created:2.0
\showcaseicon{pinterest-square}{faPinterestSquare}{2.0}% unicode:f0d3 / created:2.0
\showcaseicon{google-plus-square}{faGooglePlusSquare}{2.0}% unicode:f0d4 / created:2.0
\showcaseicon{google-plus}{faGooglePlus}{2.0}% unicode:f0d5 / created:2.0
\showcaseicon{linkedin}{faLinkedin}{2.0}% unicode:f0e1 / created:2.0
\showcaseicon{github-alt}{faGithubAlt}{3.0}% unicode:f113 / created:3.0
\showcaseicon{maxcdn}{faMaxcdn}{3.1}% unicode:f136 / created:3.1
\showcaseicon{html5}{faHtml5}{3.1}% unicode:f13b / created:3.1
\showcaseicon{css3}{faCss3}{3.1}% unicode:f13c / created:3.1
\showcaseicon{btc}{faBtc}{3.2}% unicode:f15a / created:3.2
\showcaseicon{bitcoin}{faBitcoin}{alias}
\showcaseicon{youtube-square}{faYoutubeSquare}{3.2}% unicode:f166 / created:3.2
\showcaseicon{youtube}{faYoutube}{3.2}% unicode:f167 / created:3.2
\showcaseicon{xing}{faXing}{3.2}% unicode:f168 / created:3.2
\showcaseicon{xing-square}{faXingSquare}{3.2}% unicode:f169 / created:3.2
\showcaseicon{youtube-play}{faYoutubePlay}{3.2}% unicode:f16a / created:3.2
\showcaseicon{dropbox}{faDropbox}{3.2}% unicode:f16b / created:3.2
\showcaseicon{stack-overflow}{faStackOverflow}{3.2}% unicode:f16c / created:3.2
\showcaseicon{instagram}{faInstagram}{3.2}% unicode:f16d / created:3.2
\showcaseicon{flickr}{faFlickr}{3.2}% unicode:f16e / created:3.2
\showcaseicon{adn}{faAdn}{3.2}% unicode:f170 / created:3.2
\showcaseicon{bitbucket}{faBitbucket}{3.2}% unicode:f171 / created:3.2
\showcaseicon{bitbucket-square}{faBitbucketSquare}{3.2}% unicode:f172 / created:3.2
\showcaseicon{tumblr}{faTumblr}{3.2}% unicode:f173 / created:3.2
\showcaseicon{tumblr-square}{faTumblrSquare}{3.2}% unicode:f174 / created:3.2
\showcaseicon{apple}{faApple}{3.2}% unicode:f179 / created:3.2
\showcaseicon{windows}{faWindows}{3.2}% unicode:f17a / created:3.2
\showcaseicon{android}{faAndroid}{3.2}% unicode:f17b / created:3.2
\showcaseicon{linux}{faLinux}{3.2}% unicode:f17c / created:3.2
\showcaseicon{dribbble}{faDribbble}{3.2}% unicode:f17d / created:3.2
\showcaseicon{skype}{faSkype}{3.2}% unicode:f17e / created:3.2
\showcaseicon{foursquare}{faFoursquare}{3.2}% unicode:f180 / created:3.2
\showcaseicon{trello}{faTrello}{3.2}% unicode:f181 / created:3.2
\showcaseicon{gittip}{faGittip}{3.2}% unicode:f184 / created:3.2
\showcaseicon{vk}{faVk}{3.2}% unicode:f189 / created:3.2
\showcaseicon{weibo}{faWeibo}{3.2}% unicode:f18a / created:3.2
\showcaseicon{renren}{faRenren}{3.2}% unicode:f18b / created:3.2
\showcaseicon{pagelines}{faPagelines}{4.0}% unicode:f18c / created:4.0
\showcaseicon{stack-exchange}{faStackExchange}{4.0}% unicode:f18d / created:4.0
\showcaseicon{vimeo-square}{faVimeoSquare}{4.0}% unicode:f194 / created:4.0
\showcaseicon{slack}{faSlack}{4.1}% unicode:f198 / created:4.1
\showcaseicon{wordpress}{faWordpress}{4.1}% unicode:f19a / created:4.1
\showcaseicon{openid}{faOpenid}{4.1}% unicode:f19b / created:4.1
\showcaseicon{yahoo}{faYahoo}{4.1}% unicode:f19e / created:4.1
\showcaseicon{google}{faGoogle}{4.1}% unicode:f1a0 / created:4.1
\showcaseicon{reddit}{faReddit}{4.1}% unicode:f1a1 / created:4.1
\showcaseicon{reddit-square}{faRedditSquare}{4.1}% unicode:f1a2 / created:4.1
\showcaseicon{stumbleupon-circle}{faStumbleuponCircle}{4.1}% unicode:f1a3 / created:4.1
\showcaseicon{stumbleupon}{faStumbleupon}{4.1}% unicode:f1a4 / created:4.1
\showcaseicon{delicious}{faDelicious}{4.1}% unicode:f1a5 / created:4.1
\showcaseicon{digg}{faDigg}{4.1}% unicode:f1a6 / created:4.1
\showcaseicon{pied-piper}{faPiedPiper}{4.1}% unicode:f1a7 / created:4.1
\showcaseicon{pied-piper-square}{faPiedPiperSquare}{alias}
\showcaseicon{pied-piper-alt}{faPiedPiperAlt}{4.1}% unicode:f1a8 / created:4.1
\showcaseicon{drupal}{faDrupal}{4.1}% unicode:f1a9 / created:4.1
\showcaseicon{joomla}{faJoomla}{4.1}% unicode:f1aa / created:4.1
\showcaseicon{behance}{faBehance}{4.1}% unicode:f1b4 / created:4.1
\showcaseicon{behance-square}{faBehanceSquare}{4.1}% unicode:f1b5 / created:4.1
\showcaseicon{steam}{faSteam}{4.1}% unicode:f1b6 / created:4.1
\showcaseicon{steam-square}{faSteamSquare}{4.1}% unicode:f1b7 / created:4.1
\showcaseicon{spotify}{faSpotify}{4.1}% unicode:f1bc / created:4.1
\showcaseicon{deviantart}{faDeviantart}{4.1}% unicode:f1bd / created:4.1
\showcaseicon{soundcloud}{faSoundcloud}{4.1}% unicode:f1be / created:4.1
\showcaseicon{vine}{faVine}{4.1}% unicode:f1ca / created:4.1
\showcaseicon{codepen}{faCodepen}{4.1}% unicode:f1cb / created:4.1
\showcaseicon{jsfiddle}{faJsfiddle}{4.1}% unicode:f1cc / created:4.1
\showcaseicon{rebel}{faRebel}{4.1}% unicode:f1d0 / created:4.1
\showcaseicon{ra}{faRa}{alias}
\showcaseicon{empire}{faEmpire}{4.1}% unicode:f1d1 / created:4.1
\showcaseicon{ge}{faGe}{alias}
\showcaseicon{git-square}{faGitSquare}{4.1}% unicode:f1d2 / created:4.1
\showcaseicon{git}{faGit}{4.1}% unicode:f1d3 / created:4.1
\showcaseicon{hacker-news}{faHackerNews}{4.1}% unicode:f1d4 / created:4.1
\showcaseicon{tencent-weibo}{faTencentWeibo}{4.1}% unicode:f1d5 / created:4.1
\showcaseicon{qq}{faQq}{4.1}% unicode:f1d6 / created:4.1
\showcaseicon{weixin}{faWeixin}{4.1}% unicode:f1d7 / created:4.1
\showcaseicon{wechat}{faWechat}{alias}
\showcaseicon{share-alt}{faShareAlt}{4.1}% unicode:f1e0 / created:4.1
\showcaseicon{share-alt-square}{faShareAltSquare}{4.1}% unicode:f1e1 / created:4.1
\end{showcase}
\subsection{Currency Icons}
\begin{showcase}
\showcaseicon{money}{faMoney}{2.0}% unicode:f0d6 / created:2.0
\showcaseicon{eur}{faEur}{3.2}% unicode:f153 / created:3.2
\showcaseicon{euro}{faEuro}{alias}
\showcaseicon{gbp}{faGbp}{3.2}% unicode:f154 / created:3.2
\showcaseicon{usd}{faUsd}{3.2}% unicode:f155 / created:3.2
\showcaseicon{dollar}{faDollar}{alias}
\showcaseicon{inr}{faInr}{3.2}% unicode:f156 / created:3.2
\showcaseicon{rupee}{faRupee}{alias}
\showcaseicon{jpy}{faJpy}{3.2}% unicode:f157 / created:3.2
\showcaseicon{cny}{faCny}{alias}
\showcaseicon{rmb}{faRmb}{alias}
\showcaseicon{yen}{faYen}{alias}
\showcaseicon{rub}{faRub}{4.0}% unicode:f158 / created:4.0
\showcaseicon{ruble}{faRuble}{alias}
\showcaseicon{rouble}{faRouble}{alias}
\showcaseicon{krw}{faKrw}{3.2}% unicode:f159 / created:3.2
\showcaseicon{won}{faWon}{alias}
\showcaseicon{btc}{faBtc}{3.2}% unicode:f15a / created:3.2
\showcaseicon{bitcoin}{faBitcoin}{alias}
\showcaseicon{try}{faTry}{4.0}% unicode:f195 / created:4.0
\showcaseicon{turkish-lira}{faTurkishLira}{alias}
\end{showcase}
\subsection{Medical Icons}
\begin{showcase}
\showcaseicon{user-md}{faUserMd}{2.0}% unicode:f0f0 / created:2.0
\showcaseicon{stethoscope}{faStethoscope}{3.0}% unicode:f0f1 / created:3.0
\showcaseicon{hospital-o}{faHospitalO}{3.0}% unicode:f0f8 / created:3.0
\showcaseicon{ambulance}{faAmbulance}{3.0}% unicode:f0f9 / created:3.0
\showcaseicon{medkit}{faMedkit}{3.0}% unicode:f0fa / created:3.0
\showcaseicon{h-square}{faHSquare}{3.0}% unicode:f0fd / created:3.0
\showcaseicon{plus-square}{faPlusSquare}{3.0}% unicode:f0fe / created:3.0
\showcaseicon{wheelchair}{faWheelchair}{4.0}% unicode:f193 / created:4.0
\end{showcase}


\PrintChanges
\PrintIndex
\end{document}
